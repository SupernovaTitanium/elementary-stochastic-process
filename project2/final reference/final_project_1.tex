%%%%%%%%%%%%%%%%%%%%%%%%%%%%%%%%%%%%%%%%%
% Thin Sectioned Essay
% LaTeX Template
% Version 1.0 (3/8/13)
%
% This template has been downloaded from:
% http://www.LaTeXTemplates.com
%
% Original Author:
% Nicolas Diaz (nsdiaz@uc.cl) with extensive modifications by:
% Vel (vel@latextemplates.com)
%
% License:
% CC BY-NC-SA 3.0 (http://creativecommons.org/licenses/by-nc-sa/3.0/)
%
%%%%%%%%%%%%%%%%%%%%%%%%%%%%%%%%%%%%%%%%%

%----------------------------------------------------------------------------------------
%	PACKAGES AND OTHER DOCUMENT CONFIGURATIONS
%----------------------------------------------------------------------------------------

\documentclass[a4paper, 11pt]{article} % Font size (can be 10pt, 11pt or 12pt) and paper size (remove a4paper for US letter paper)
\usepackage{listings}   
\usepackage{amsmath}
\usepackage{wrapfig} % Allows in-line images
\usepackage{graphicx}
\usepackage{mathpazo} % Use the Palatino font
\usepackage[T1]{fontenc} % Required for accented characters
\usepackage{floatrow}
\usepackage{lmodern}
\usepackage{graphicx}
\usepackage{amsthm}
\usepackage{float}
\newtheorem*{theorem*}{Theorem}
\newtheorem*{corollary*}{Corollary}
\usepackage{lingmacros}
\usepackage{subfiles}
\usepackage{mathtools}
\linespread{1.05} % Change line spacing here, Palatino benefits from a slight increase by default
\newtheorem{theorem}{Theorem}
\newtheorem{observation}[theorem]{Observation}
\newtheorem*{remark}{Remark}
\newtheorem*{definition}{Definition}
\newtheorem*{property}{Property}
\newtheorem*{goal}{Goal}
\makeatletter
\renewcommand\@biblabel[1]{\textbf{#1.}} % Change the square brackets for each bibliography item from '[1]' to '1.'
\renewcommand{\@listI}{\itemsep=0pt} % Reduce the space between items in the itemize and enumerate environments and the bibliography

\renewcommand{\maketitle}{ % Customize the title - do not edit title and author name here, see the TITLE block below
\begin{flushright} % Right align
{\LARGE\@title} % Increase the font size of the title

\vspace{50pt} % Some vertical space between the title and author name

{\large\@author} % Author name
\\\@date % Date

\vspace{40pt} % Some vertical space between the author block and abstract
\end{flushright}
}

%----------------------------------------------------------------------------------------
%	TITLE
%----------------------------------------------------------------------------------------

\title{\textbf{Limiting Theorem of Empirical Process: An Example in Kolmogorov-Smirov Test and Brownian Bridge}\\ % Title
Stochostic Process: Final Project} % Subtitle

\author{\textsc{Wei-Chang Lee, Chi-Ning Chou} % Author
\\{\textit{PROFESSOR RAOUL NORMAND}}} % Institution

\date{\today} % Date

%----------------------------------------------------------------------------------------

\begin{document}

\maketitle % Print the title section

%----------------------------------------------------------------------------------------
%	ABSTRACT AND KEYWORDS
%----------------------------------------------------------------------------------------

%\renewcommand{\abstractname}{Summary} % Uncomment to change the name of the abstract to something else

\begin{abstract}
In traditional statistical setting, we have the nice LLN and CLT, which give us an asymptotic sense in Gaussian distribution about the convergence behaviour of the statistics. When it comes to empirical process, however, the quadratic variation term breaks the beautiful structure and thus we have to come up with stronger sense about the asymptotic behaviour of the empirical process.

Kolmogorov statistics, which is the infinity norm of the empirical process, can tell us something about the asymptotic behaviour of the empirical process while it is the uniform upper bound of the whole process. With it's nice {\it distribution free} property and two strong theorem: Glivenko-Cantelli Theorem and Donsker Theorem, the uniform LLN and uniform CLT is guaranteed. Furthermore, the asymptotic behaviour of the Kolmogorov statistics will converge to the supremum norm of a special Gaussian process: Brownian Bridge. This is the beautiful analogy of asymptotic convergence to Gaussian random variable of traditional setting.

In the first two section, we will introduce the two limiting theorem of empirical process and the Gaussian process respectively. Then, the last section will prove the uniform CLT of empirical process and give a short tutorial on Kolmogorov test. 


\end{abstract}

\hspace*{5,6mm}\textit{Keywords:} Empirical Process, Glivenko-Cantelli Theorem, Donsker Theorem, Gaussian Process, Brownian Bridge, Kolmogorov-Smirov test % Keywords

\vspace{30pt} % Some vertical space between the abstract and first section

%----------------------------------------------------------------------------------------
%	ESSAY BODY
%----------------------------------------------------------------------------------------

% outline
\tableofcontents
\setcounter{tocdepth}{1}

%----------------------------------------------------------------------------------------
%	Section 1: EMpirical Processs Theory
%----------------------------------------------------------------------------------------
\section{Empirical Process Theory}
\subfile{section1_empirical_process_theory}

%----------------------------------------------------------------------------------------
%	Section 2: Gaussian Process
%----------------------------------------------------------------------------------------
\section{Gaussian Process}
\subfile{section2_gaussian_process}


%----------------------------------------------------------------------------------------
%	Section 3: Kolmogorov-Smirov Test
%----------------------------------------------------------------------------------------
\section{Donsker Theorem and Kolmogorov-Smirov Test}
\subfile{section3_kolmogorov_smirov_test}


\section{Appendix}

It's welcome to discuss the code with us!
%------------------------------------------------


\bibliographystyle{plain}
\begin{thebibliography}{9}

\bibitem{BB}
\emph{Brownian Bridge},
http://www.math.uah.edu/stat/brown/Bridge.html

\end{thebibliography}



\end{document}