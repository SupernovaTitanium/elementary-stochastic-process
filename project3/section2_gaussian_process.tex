\documentclass[final_project_1.tex]{subfiles}
\begin{document}
\subsection{Properties}
\begin{definition}[Gaussian Process]
A Gaussuan process $\{X_t,t\in T\}$ is a stochastic process that any finite linear combination of samples has a joint Gaussian distribution.
\end{definition}

\paragraph{}
The most common example is the Brownian motion $\{B_t,t\in T\}$. The variance of $B_t$ is $t$ and the covariance between $B_t$ and $B_s$ is $t\wedge s$.

\paragraph{}
One of the most important property of Gaussian process is that a the behaviour of a Gaussian process is fully determined by its covariance function.
\begin{property}
The covariance function of a Gaussian process completely determine its behaviour.
\end{property}
This property is very important. In the proof of Donsker theorem, we will apply this result to identify the behaviour of the statistics that we are desired.

As a remark, to prove this property, you can consider the characteristic function of the Gaussian process then you will find something interesting.

\subsection{Brownian Bridge}
\begin{theorem}
A Brownian Bridge is a Gaussian process with covariance function	
\end{theorem}


\end{document}