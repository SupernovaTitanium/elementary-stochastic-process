\documentclass[final_project_1.tex]{subfiles}
\begin{document}

\subsection{Properties}

\begin{definition}[Gaussian Process]
A Gaussuan process $\{X_t,t\in T\}$ is a stochastic process that any finite linear combination of samples has a joint Gaussian distribution.
\end{definition}
The most common example is the Brownian motion $\{B_t,t\in T\}$. It starts from 0 with probability 1. The variance is $t$ and the covariance between $B_t$ and $B_s$ is $t\wedge s$.
\paragraph{}
One of the most important property of Gaussian process is that a the behaviour of a Gaussian process is fully determined by its mean and covariance function.
\begin{property}[Uniqueness Property]
The covariance function of a Gaussian process completely determine its behaviour.
\end{property}
In the proof of Donsker theorem, we will apply this result to identify the behaviour of the statistics that we are desired. As a little remark, to prove this property, you can consider the characteristic function of the Gaussian process then you will find something interesting.

\subsection{Brownian Bridge}

\paragraph{}
Brownian bridge is one the the common Gaussian process in real life. It is obtained by taking a standard Brownian motion restricted on [0, 1] and conditioning on the event that $X_0 = X_1=0$. This definition gives us an intuition about the distribution of a Brownian bridge. First, we can see that the future of the Brownian bridge is not independent to the past. Moreover, with this operational definition, we can have a picture about why this process is called {\it bridge}.

Before introducing you more properties about Brownian bridge, let's take a look at its formal definition in advance.

\begin{definition}
A Brownian Bridge is a Gaussian process with covariance function	$Cov(X_s,X_t)=s\wedge t-st$.
\end{definition}

\paragraph{}
As you can see, Brownian bridge is strongly related to Brownian motion. And actually, they can generate each other!That is, given a Brownian motion, then you can construct a Brownian bridge and vice versa.

In the following we will list some constructions of Brownian bridge via Brownian motion without proving it since they are not very important for our goal: Donsker theorem. 

\begin{property}[Construction of Brownian Bridge]
Suppose $\mathbf{Z} = \{Z_t: t\in[0,1]\}$ is a standard Brownian motion on [0,1]. Then $\mathbf{X} = \{X_t:t\in [0,1]\}$ is a Brownian bridge if one of the following is true
\begin{enumerate}
\item $X_t = Z_t-tZ_1$ \\
\item $X_t = (1-t)Z(\frac{t}{1-t})$ \\
\item $X_t = (1-t)\int^t_0 \frac{1}{1-s}dZ_s$ \\
\item $dX_t=\frac{X_t}{1-t}dt + dZ_t$
\end{enumerate}
\end{property}
To verify the above constructions, you can simply check that the covariance function of $X_t$ and $X_s$ is $t\wedge s - ts$.

\paragraph{}
Next, let's consider the distribution of the supremum of Brownian bridge.
\begin{property}[Supremum of Brownian bridge]
Let $\mathbf{X}=\{X_t:t\in[0,1]\}$ be a Brownian bridge and $S^+ = \sup_{0\leq t\leq 1}X(t)$ and $S = \sup_{0\leq t\leq 1}|X(t)|$. Then,
\begin{enumerate}
\item $F(S^+ \leq x) = 1-\exp(-2x^2)$\\
\item $F(S\leq x) = 1 + 2\sum^{\infty}_{k=1}(-1)^k\exp(-2k^2x^2)$
\end{enumerate}
\end{property}

\end{document}